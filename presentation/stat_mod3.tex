% Options for packages loaded elsewhere
\PassOptionsToPackage{unicode}{hyperref}
\PassOptionsToPackage{hyphens}{url}
\PassOptionsToPackage{dvipsnames,svgnames,x11names}{xcolor}
%
\documentclass[
  12pt]{article}

\usepackage{amsmath,amssymb}
\usepackage{lmodern}
\usepackage{setspace}
\usepackage{iftex}
\ifPDFTeX
  \usepackage[T1]{fontenc}
  \usepackage[utf8]{inputenc}
  \usepackage{textcomp} % provide euro and other symbols
\else % if luatex or xetex
  \usepackage{unicode-math}
  \defaultfontfeatures{Scale=MatchLowercase}
  \defaultfontfeatures[\rmfamily]{Ligatures=TeX,Scale=1}
\fi
% Use upquote if available, for straight quotes in verbatim environments
\IfFileExists{upquote.sty}{\usepackage{upquote}}{}
\IfFileExists{microtype.sty}{% use microtype if available
  \usepackage[]{microtype}
  \UseMicrotypeSet[protrusion]{basicmath} % disable protrusion for tt fonts
}{}
\makeatletter
\@ifundefined{KOMAClassName}{% if non-KOMA class
  \IfFileExists{parskip.sty}{%
    \usepackage{parskip}
  }{% else
    \setlength{\parindent}{0pt}
    \setlength{\parskip}{6pt plus 2pt minus 1pt}}
}{% if KOMA class
  \KOMAoptions{parskip=half}}
\makeatother
\usepackage{xcolor}
\setlength{\emergencystretch}{3em} % prevent overfull lines
\setcounter{secnumdepth}{5}
% Make \paragraph and \subparagraph free-standing
\ifx\paragraph\undefined\else
  \let\oldparagraph\paragraph
  \renewcommand{\paragraph}[1]{\oldparagraph{#1}\mbox{}}
\fi
\ifx\subparagraph\undefined\else
  \let\oldsubparagraph\subparagraph
  \renewcommand{\subparagraph}[1]{\oldsubparagraph{#1}\mbox{}}
\fi


\providecommand{\tightlist}{%
  \setlength{\itemsep}{0pt}\setlength{\parskip}{0pt}}\usepackage{longtable,booktabs,array}
\usepackage{calc} % for calculating minipage widths
% Correct order of tables after \paragraph or \subparagraph
\usepackage{etoolbox}
\makeatletter
\patchcmd\longtable{\par}{\if@noskipsec\mbox{}\fi\par}{}{}
\makeatother
% Allow footnotes in longtable head/foot
\IfFileExists{footnotehyper.sty}{\usepackage{footnotehyper}}{\usepackage{footnote}}
\makesavenoteenv{longtable}
\usepackage{graphicx}
\makeatletter
\def\maxwidth{\ifdim\Gin@nat@width>\linewidth\linewidth\else\Gin@nat@width\fi}
\def\maxheight{\ifdim\Gin@nat@height>\textheight\textheight\else\Gin@nat@height\fi}
\makeatother
% Scale images if necessary, so that they will not overflow the page
% margins by default, and it is still possible to overwrite the defaults
% using explicit options in \includegraphics[width, height, ...]{}
\setkeys{Gin}{width=\maxwidth,height=\maxheight,keepaspectratio}
% Set default figure placement to htbp
\makeatletter
\def\fps@figure{htbp}
\makeatother

\addtolength{\oddsidemargin}{-.5in}%
\addtolength{\evensidemargin}{-1in}%
\addtolength{\textwidth}{1in}%
\addtolength{\textheight}{1.7in}%
\addtolength{\topmargin}{-1in}%
\makeatletter
\makeatother
\makeatletter
\makeatother
\makeatletter
\@ifpackageloaded{caption}{}{\usepackage{caption}}
\AtBeginDocument{%
\ifdefined\contentsname
  \renewcommand*\contentsname{Table of contents}
\else
  \newcommand\contentsname{Table of contents}
\fi
\ifdefined\listfigurename
  \renewcommand*\listfigurename{List of Figures}
\else
  \newcommand\listfigurename{List of Figures}
\fi
\ifdefined\listtablename
  \renewcommand*\listtablename{List of Tables}
\else
  \newcommand\listtablename{List of Tables}
\fi
\ifdefined\figurename
  \renewcommand*\figurename{Figure}
\else
  \newcommand\figurename{Figure}
\fi
\ifdefined\tablename
  \renewcommand*\tablename{Table}
\else
  \newcommand\tablename{Table}
\fi
}
\@ifpackageloaded{float}{}{\usepackage{float}}
\floatstyle{ruled}
\@ifundefined{c@chapter}{\newfloat{codelisting}{h}{lop}}{\newfloat{codelisting}{h}{lop}[chapter]}
\floatname{codelisting}{Listing}
\newcommand*\listoflistings{\listof{codelisting}{List of Listings}}
\makeatother
\makeatletter
\@ifpackageloaded{caption}{}{\usepackage{caption}}
\@ifpackageloaded{subcaption}{}{\usepackage{subcaption}}
\makeatother
\makeatletter
\@ifpackageloaded{tcolorbox}{}{\usepackage[many]{tcolorbox}}
\makeatother
\makeatletter
\@ifundefined{shadecolor}{\definecolor{shadecolor}{rgb}{.97, .97, .97}}
\makeatother
\makeatletter
\makeatother
\ifLuaTeX
  \usepackage{selnolig}  % disable illegal ligatures
\fi
\usepackage[]{natbib}
\bibliographystyle{agsm}
\IfFileExists{bookmark.sty}{\usepackage{bookmark}}{\usepackage{hyperref}}
\IfFileExists{xurl.sty}{\usepackage{xurl}}{} % add URL line breaks if available
\urlstyle{same} % disable monospaced font for URLs
\hypersetup{
  pdfauthor={Marco Zanotti},
  pdfkeywords={propensity score matching, time series, exchange rate
regimes, economic growth},
  colorlinks=true,
  linkcolor={blue},
  filecolor={Maroon},
  citecolor={Blue},
  urlcolor={Blue},
  pdfcreator={LaTeX via pandoc}}


\begin{document}


\def\spacingset#1{\renewcommand{\baselinestretch}%
{#1}\small\normalsize} \spacingset{1}


%%%%%%%%%%%%%%%%%%%%%%%%%%%%%%%%%%%%%%%%%%%%%%%%%%%%%%%%%%%%%%%%%%%%%%%%%%%%%%

\date{October 24, 2023}
\title{\bf Research Review

The effects of exchange rate regimes on economic growth: evidence from
propensity score matching estimates.}
\author{
Marco Zanotti\\
University of Milano Bicocca\\
}
\maketitle

\bigskip
\bigskip
\begin{abstract}
After a brief summary, this review points out the main advantages and
the main drawbacks of the research.
\end{abstract}

\noindent%
{\it Keywords:} propensity score matching, time series, exchange rate
regimes, economic growth
\vfill

\newpage
\spacingset{1.9} % DON'T change the spacing!
\ifdefined\Shaded\renewenvironment{Shaded}{\begin{tcolorbox}[frame hidden, sharp corners, borderline west={3pt}{0pt}{shadecolor}, breakable, enhanced, boxrule=0pt, interior hidden]}{\end{tcolorbox}}\fi

\setstretch{1}
\hypertarget{summary}{%
\section{Summary}\label{summary}}

The paper focuses on the impact of different exchange rate regimes on
economic growth and employs a variety of non-parametric matching methods
to address potential biases in previous research. The key points and
findings of the paper can be summarized as follows:

\begin{enumerate}
\def\labelenumi{\arabic{enumi}.}
\item
  Background: The paper discusses the ongoing debate in the
  macroeconomic literature about how different exchange rate regimes
  affect economic growth. Previous studies using linear regression
  models have provided mixed results, possibly due to self-selection
  bias and other issues.
\item
  Methodology: The paper introduces a new approach using matching
  techniques to estimate the Average Treatment Effect (ATT) of exchange
  rate regimes on economic growth. These methods aim to address
  self-selection bias and other potential sources of bias.
\item
  Data: The study uses a comprehensive dataset covering 164 countries
  from 1970 to 2007, considering the period after the collapse of the
  Bretton Woods fixed exchange rate system. The key variables include
  exchange rate regimes (fixed or floating), economic growth rates, and
  several control variables like gross capital formation rate, trade as
  a share of GDP, industrialization rate, inflation rate, exports and
  imports as a share of GDP, and foreign direct investment as a share of
  GDP.
\item
  Propensity Score Matching: The paper employs propensity score matching
  to estimate the effect of exchange rate regimes on economic growth.
  This approach involves creating a probability score for each country
  to determine the likelihood of choosing a floating or fixed exchange
  rate regime based on observable covariates.
\item
  Matching Methods: The study uses various matching methods, including
  nearest-neighbor matching, radius matching, kernel matching, and local
  linear matching, to find suitable matches between countries with
  different exchange rate regimes.
\item
  Common Support: The research ensures that there is good overlap
  (common support) between the control group (fixed exchange rate
  regime) and the treatment group (floating exchange rate regime). This
  common support ensures that observed characteristics of treatment
  group countries can also be found among control group countries.
\item
  Findings: The results of all matching methods consistently indicate
  that the average treatment effect of floating exchange rate regimes on
  economic growth is statistically insignificant. In other words, the
  study finds no evidence to suggest that adopting a floating exchange
  rate regime leads to higher economic growth compared to a fixed
  exchange rate regime.
\item
  Robustness Analysis: The paper conducts a robustness analysis using
  Rosenbaum's bounds to account for potential hidden biases. The results
  remain strong and robust, supporting the conclusion that exchange rate
  regime choice does not significantly impact economic growth.
\end{enumerate}

In summary, this statistical paper employs innovative matching
techniques to examine the effect of exchange rate regimes on economic
growth, finding that there is no statistically significant difference in
economic growth between countries with floating and fixed exchange rate
regimes.

\hypertarget{main-advantages}{%
\section{Main Advantages}\label{main-advantages}}

\begin{enumerate}
\def\labelenumi{\arabic{enumi}.}
\item
  Innovative Methodology: One of the significant advantages of the paper
  is its innovative approach to examining the impact of exchange rate
  regimes on economic growth. By using propensity score matching and a
  variety of matching techniques, the paper addresses potential biases
  that may have affected previous research.
\item
  Comprehensive Data: The use of a comprehensive dataset covering 164
  countries over a substantial time period provides a robust foundation
  for the analysis. The inclusion of various control variables adds
  depth to the analysis and helps control for potential confounding
  factors.
\item
  Robustness Analysis: The paper conducts a robustness analysis using
  Rosenbaum's bounds to assess the influence of hidden biases. This
  approach adds credibility to the findings and strengthens the
  conclusion that exchange rate regimes do not significantly affect
  economic growth.
\item
  Consistency in Results: The paper reports consistent results across
  different matching methods, enhancing the reliability of the findings.
  The fact that various techniques yield similar conclusions adds to the
  paper's credibility.
\end{enumerate}

\hypertarget{main-drawbacks}{%
\section{Main Drawbacks}\label{main-drawbacks}}

\begin{enumerate}
\def\labelenumi{\arabic{enumi}.}
\item
  Data Limitations: While the paper utilizes a comprehensive dataset,
  economic data can be subject to various limitations, such as
  measurement errors and reporting biases. It's important to acknowledge
  the potential limitations of the data source and discuss their
  implications.
\item
  Assumption of Common Support: The paper relies on the assumption of
  common support between control and treatment groups. While this is a
  crucial assumption in propensity score matching, the paper should
  provide a robustness analysis to explore the sensitivity of results to
  variations in this assumption.
\item
  Limited Causality Exploration: The paper primarily focuses on
  estimating the association between exchange rate regimes and economic
  growth. It may benefit from discussing potential mechanisms or
  channels through which exchange rate regimes might affect growth,
  providing more insights into the economic theory behind the findings.
\item
  Extrapolation of Findings: The paper concludes that there is no
  significant impact of exchange rate regimes on economic growth.
  However, it's essential to recognize that the findings may not apply
  universally across all time periods, countries, or economic contexts.
  Some discussion of the potential limitations of generalizing these
  findings could enhance the paper's completeness.
\item
  Policy Implications: The paper could benefit from a discussion of the
  policy implications of its findings. While it suggests that exchange
  rate regimes do not significantly affect growth, policymakers might
  still be interested in understanding the broader economic implications
  of different exchange rate policies.
\end{enumerate}

\hypertarget{conclusion}{%
\section{Conclusion}\label{conclusion}}

In conclusion, the paper introduces an innovative methodology and uses a
comprehensive dataset to examine the impact of exchange rate regimes on
economic growth. However, it is essential to acknowledge data
limitations, assumptions made in the analysis, and potential
generalizability issues. Overall, the paper makes a valuable
contribution to the empirical literature on exchange rate regimes and
economic growth but should be interpreted within the context of its
limitations.



\end{document}
