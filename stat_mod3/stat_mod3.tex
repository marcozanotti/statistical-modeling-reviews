% Options for packages loaded elsewhere
\PassOptionsToPackage{unicode}{hyperref}
\PassOptionsToPackage{hyphens}{url}
\PassOptionsToPackage{dvipsnames,svgnames,x11names}{xcolor}
%
\documentclass[
  12pt]{article}

\usepackage{amsmath,amssymb}
\usepackage{lmodern}
\usepackage{setspace}
\usepackage{iftex}
\ifPDFTeX
  \usepackage[T1]{fontenc}
  \usepackage[utf8]{inputenc}
  \usepackage{textcomp} % provide euro and other symbols
\else % if luatex or xetex
  \usepackage{unicode-math}
  \defaultfontfeatures{Scale=MatchLowercase}
  \defaultfontfeatures[\rmfamily]{Ligatures=TeX,Scale=1}
\fi
% Use upquote if available, for straight quotes in verbatim environments
\IfFileExists{upquote.sty}{\usepackage{upquote}}{}
\IfFileExists{microtype.sty}{% use microtype if available
  \usepackage[]{microtype}
  \UseMicrotypeSet[protrusion]{basicmath} % disable protrusion for tt fonts
}{}
\makeatletter
\@ifundefined{KOMAClassName}{% if non-KOMA class
  \IfFileExists{parskip.sty}{%
    \usepackage{parskip}
  }{% else
    \setlength{\parindent}{0pt}
    \setlength{\parskip}{6pt plus 2pt minus 1pt}}
}{% if KOMA class
  \KOMAoptions{parskip=half}}
\makeatother
\usepackage{xcolor}
\setlength{\emergencystretch}{3em} % prevent overfull lines
\setcounter{secnumdepth}{5}
% Make \paragraph and \subparagraph free-standing
\ifx\paragraph\undefined\else
  \let\oldparagraph\paragraph
  \renewcommand{\paragraph}[1]{\oldparagraph{#1}\mbox{}}
\fi
\ifx\subparagraph\undefined\else
  \let\oldsubparagraph\subparagraph
  \renewcommand{\subparagraph}[1]{\oldsubparagraph{#1}\mbox{}}
\fi


\providecommand{\tightlist}{%
  \setlength{\itemsep}{0pt}\setlength{\parskip}{0pt}}\usepackage{longtable,booktabs,array}
\usepackage{calc} % for calculating minipage widths
% Correct order of tables after \paragraph or \subparagraph
\usepackage{etoolbox}
\makeatletter
\patchcmd\longtable{\par}{\if@noskipsec\mbox{}\fi\par}{}{}
\makeatother
% Allow footnotes in longtable head/foot
\IfFileExists{footnotehyper.sty}{\usepackage{footnotehyper}}{\usepackage{footnote}}
\makesavenoteenv{longtable}
\usepackage{graphicx}
\makeatletter
\def\maxwidth{\ifdim\Gin@nat@width>\linewidth\linewidth\else\Gin@nat@width\fi}
\def\maxheight{\ifdim\Gin@nat@height>\textheight\textheight\else\Gin@nat@height\fi}
\makeatother
% Scale images if necessary, so that they will not overflow the page
% margins by default, and it is still possible to overwrite the defaults
% using explicit options in \includegraphics[width, height, ...]{}
\setkeys{Gin}{width=\maxwidth,height=\maxheight,keepaspectratio}
% Set default figure placement to htbp
\makeatletter
\def\fps@figure{htbp}
\makeatother

\addtolength{\oddsidemargin}{-.5in}%
\addtolength{\evensidemargin}{-1in}%
\addtolength{\textwidth}{1in}%
\addtolength{\textheight}{1.7in}%
\addtolength{\topmargin}{-1in}%
\makeatletter
\makeatother
\makeatletter
\makeatother
\makeatletter
\@ifpackageloaded{caption}{}{\usepackage{caption}}
\AtBeginDocument{%
\ifdefined\contentsname
  \renewcommand*\contentsname{Table of contents}
\else
  \newcommand\contentsname{Table of contents}
\fi
\ifdefined\listfigurename
  \renewcommand*\listfigurename{List of Figures}
\else
  \newcommand\listfigurename{List of Figures}
\fi
\ifdefined\listtablename
  \renewcommand*\listtablename{List of Tables}
\else
  \newcommand\listtablename{List of Tables}
\fi
\ifdefined\figurename
  \renewcommand*\figurename{Figure}
\else
  \newcommand\figurename{Figure}
\fi
\ifdefined\tablename
  \renewcommand*\tablename{Table}
\else
  \newcommand\tablename{Table}
\fi
}
\@ifpackageloaded{float}{}{\usepackage{float}}
\floatstyle{ruled}
\@ifundefined{c@chapter}{\newfloat{codelisting}{h}{lop}}{\newfloat{codelisting}{h}{lop}[chapter]}
\floatname{codelisting}{Listing}
\newcommand*\listoflistings{\listof{codelisting}{List of Listings}}
\makeatother
\makeatletter
\@ifpackageloaded{caption}{}{\usepackage{caption}}
\@ifpackageloaded{subcaption}{}{\usepackage{subcaption}}
\makeatother
\makeatletter
\@ifpackageloaded{tcolorbox}{}{\usepackage[many]{tcolorbox}}
\makeatother
\makeatletter
\@ifundefined{shadecolor}{\definecolor{shadecolor}{rgb}{.97, .97, .97}}
\makeatother
\makeatletter
\makeatother
\ifLuaTeX
  \usepackage{selnolig}  % disable illegal ligatures
\fi
\usepackage[]{natbib}
\bibliographystyle{agsm}
\IfFileExists{bookmark.sty}{\usepackage{bookmark}}{\usepackage{hyperref}}
\IfFileExists{xurl.sty}{\usepackage{xurl}}{} % add URL line breaks if available
\urlstyle{same} % disable monospaced font for URLs
\hypersetup{
  pdfauthor={Marco Zanotti},
  pdfkeywords={propensity score matching, time series, exchange rate
regimes, economic growth},
  colorlinks=true,
  linkcolor={blue},
  filecolor={Maroon},
  citecolor={Blue},
  urlcolor={Blue},
  pdfcreator={LaTeX via pandoc}}


\begin{document}


\def\spacingset#1{\renewcommand{\baselinestretch}%
{#1}\small\normalsize} \spacingset{1}


%%%%%%%%%%%%%%%%%%%%%%%%%%%%%%%%%%%%%%%%%%%%%%%%%%%%%%%%%%%%%%%%%%%%%%%%%%%%%%

\date{October 24, 2023}
\title{\bf Research Review

The effects of exchange rate regimes on economic growth: evidence from
propensity score matching estimates.}
\author{
Marco Zanotti\\
University of Milano Bicocca\\
}
\maketitle

\bigskip
\bigskip
\begin{abstract}
After a brief summary, this review points out the main advantages and
the main drawbacks of the research.
\end{abstract}

\noindent%
{\it Keywords:} propensity score matching, time series, exchange rate
regimes, economic growth
\vfill

\newpage
\spacingset{1.9} % DON'T change the spacing!
\ifdefined\Shaded\renewenvironment{Shaded}{\begin{tcolorbox}[interior hidden, sharp corners, borderline west={3pt}{0pt}{shadecolor}, breakable, enhanced, frame hidden, boxrule=0pt]}{\end{tcolorbox}}\fi

\setstretch{1.5}
\hypertarget{summary}{%
\section{Summary}\label{summary}}

The paper focuses on the impact of different exchange rate regimes on
economic growth and employs a variety of non-parametric matching methods
to address potential biases in previous research. For this purpose a
comprehensive dataset covering 164 countries from 1970 to 2007,
considering the period after the collapse of the Bretton Woods fixed
exchange rate system, and many control variables. Whereas previous
studies used mainly linear regression models, likely introducing
self-selection bias and other issues, the authors adopted matching
techniques to estimate the Average Treatment Effect (ATE) of exchange
rate regimes on economic growth, aiming to address self-selection
bias.\\
In this study, propensity score is employed to estimate the effect of
exchange rate regimes on economic growth. This approach involves
creating a probability score for each country to determine the
likelihood of choosing a floating or fixed exchange rate regime based on
observable covariates. Moreover, various matching methods, including
nearest-neighbor matching, radius matching, kernel matching, and local
linear matching, are tested to find suitable matches between countries
with different exchange rate regimes.\\
The results of all matching methods consistently indicate that the
average treatment effect of floating exchange rate regimes on economic
growth is statistically insignificant. In other words, the study finds
no evidence to suggest that adopting a floating exchange rate regime
leads to higher economic growth compared to a fixed exchange rate
regime.

\hypertarget{advantages-and-limitations}{%
\section{Advantages and Limitations}\label{advantages-and-limitations}}

The authors used an innovative methodology for the research area since
by using propensity score and a variety of matching techniques, they
addressed potential biases that may have affected previous researches.
Moreover, the use of a comprehensive dataset covering 164 countries over
a substantial time period provides a robust foundation for the analysis
and the inclusion of various control variables helps control for
potential confounding factors.\\
Through a common support analysis, the research ensures that there is
good overlap between the control group (fixed exchange rate regime) and
the treatment group (floating exchange rate regime). This common support
ensures that observed characteristics of treatment group countries can
also be found among control group countries.\\
The SUTVA assumptions is satisfied by the pre and post Bretton Woods
regimes.\\
Finally, The paper reports consistent results across different matching
methods and conducts also a robustness analysis using Rosenbaum's bounds
to assess the influence of hidden biases. This approach adds credibility
to the findings and strengthens the conclusion that exchange rate
regimes do not significantly affect economic growth.

On the limitation side, instead, the paper relies on the assumption of
common support between control and treatment groups. While this is a
crucial assumption in propensity score matching, the paper should
provide some check analysis to explore the sensitivity of results to
variations in this assumption.\\
Moreover, the authors are dealing with a panel dataset, hence the
classic matching methods used may not be very appropriate since they
lack to consider the time dimension. For instance, one may refine the
matching first selecting a set of control observations from other units
in the same time period that have an identical treatment history for a
specified time span so that matched control observations become similar
to the treated observation in terms of covariate history. Furthermore,
it is not clear whether the authors used a multilevel model to estimate
the propensity score. This kind of approach could improve significantly
the estimation since it considers also the time variability of the
covariates. Finally, the paper concludes that there is no significant
impact of exchange rate regimes on economic growth. However, it's
essential to recognize that the findings may not apply universally
across all time periods, countries, or economic contexts. Some
discussion of the potential limitations of generalizing these findings
could enhance the paper's completeness.

\hypertarget{conclusion}{%
\section{Conclusion}\label{conclusion}}

In conclusion, the paper introduces an innovative methodology and uses a
comprehensive dataset to examine the impact of exchange rate regimes on
economic growth. Overall, the paper makes a valuable contribution to the
empirical literature on exchange rate regimes and economic growth but
should be interpreted within the context of its limitations.



\end{document}
